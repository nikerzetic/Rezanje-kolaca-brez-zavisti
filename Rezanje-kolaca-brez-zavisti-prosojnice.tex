% !TeX spellcheck = sl_SI
\documentclass{beamer}
\usepackage[slovene]{babel}
\usepackage[utf8]{inputenc}
\usepackage[T1]{fontenc}
\usepackage{lmodern}

\usepackage{graphicx}
\usepackage{float}

\usepackage{amssymb}
\usepackage{amsmath,amsfonts}
\usepackage{amsthm}
\usepackage{mathtools}
\usepackage{subfigure}
\usepackage{wrapfig}
\usepackage{lipsum}
\usepackage{tikz}
\usepackage{multirow}

\usepackage{array}
\usepackage{multicol}
\usepackage{enumerate}

\usepackage{booktabs}

\usetheme{Berlin}
\usecolortheme{dolphin}
%\useinnertheme[shadows]{rounded}
%\useoutertheme{infolines}
\beamertemplatenavigationsymbolsempty
\setbeamertemplate{navigation sybols}{}

\newtheorem{definicija}{Definicija}
\newtheorem{izrek}{Izrek}

\title{\textbf{Rezanje kolača brez zavisti}}
\subtitle{Predstavitev protokolov razdeljevanja}
\author{Nik Erzetič}

\institute[FMF]{\centering \includegraphics[width=0.4\linewidth]{UL-Fakulteta-za-matematiko-in-fiziko.jpg}}
\date{}

\begin{document}
	
	\begin{frame}[plain]
		\titlepage
	\end{frame}

	\section{Uvod}
	
	\subsection{Kazalo}
	
	\begin{frame}
		\frametitle{Kazalo}
		\tableofcontents[pausesections]
	\end{frame}

	\section{Protokoli}
	
	\subsection{Razreži in razdeli}
	
	\begin{frame}
		\frametitle{Razreži in razdeli}
		\begin{enumerate}
			\item Igralec 1 kolač razreže na 2 dela.
			
			\item Igralec 2 izbere kos.
			
			\item Igralec 1 dobi preostali kos.
		\end{enumerate}
	\end{frame}

	\subsection{Proporcionalni protok za $n = 3$}
	
	\begin{frame}
		\frametitle{Proporcionalni protok za $n = 3$}
		\begin{enumerate}
			\item Igralec 1 razreže kolač na 3 dele.
			
			\item Igralec 2 ne stori nič ali označi 2 kosa.
			
			\item[] Če igralec 2 ne stori nič:
			
			\setcounter{enumi}{2}
			
			\item \qquad Igralec 3 izbere kos.
			
			\item \qquad Igralec 2 izbere kos.
			
			\item \qquad Igralec 1 dobi preostali kos.
			
			\item[] Če igralec 2 označi 2 kosa:
			
			\setcounter{enumi}{2}
			
			\item \qquad Igralec 3 ne stori nič ali označi 2 kosa.
			
			\item[] \qquad Če igralec 3 ne stori nič:
			
			\setcounter{enumi}{3}
			
			\item \qquad \qquad Igralec 2 izbere kos.
		\end{enumerate}
	\end{frame}
	
	\begin{frame}
		\begin{enumerate}			
			\setcounter{enumi}{4}
			
			\item \qquad \qquad Igralec 3 izbere kos.
			
			\item \qquad \qquad Igralec 1 dobi preostali kos.
			
			\item[] \qquad Če igralec 3 označi 2 kosa:
			
			\setcounter{enumi}{3}
			
			\item \qquad \qquad Igralec 1 izbere kos, ki sta ga označila igralec 2 in igralec 3.
			
			\item \qquad \qquad Preostala kosa se združita v nov kolač.
			
			\item \qquad \qquad Protokol \textbf{razreži in izberi} med igralcem 2 in igralcem 3.
		\end{enumerate}
	\end{frame}

	\subsection{Proporcionalni protokol za poljuben $n$}

	\begin{frame}
		\frametitle{Proporcionalni protokol za poljuben $n$}
		\begin{enumerate}
			\item Igralec 1 odreže kos od kolača.
			
			\item \begin{itemize}
				
				\item Igralec 2 ne stori nič ali obreže odrezani kos.
				
				\item[] $\vdots$
				
				\item Igralec i ne stori nič ali obreže odrezani kos.
				
				\item[] $\vdots$
				
				\item Igralec n ne stori nič ali obreže odrezani kos.
				
			\end{itemize}
			
			\item Zadnji igralec, ki je obrezal kos, ali igralec 1, če nihče ni obrezal kosa, prejme ta kos. 
			
			\item Odrezke se združijo s kolačem.
			
			\item Koraki 1. - 4. se ponavljajo, dokler ne ostaneta dva igralca.
			
			\item Protokol \textbf{razreži in izberi} med preostalima igralcem.
		\end{enumerate}
	\end{frame}

	\subsection{Protokol brez zavisti za $n = 3$}

	\begin{frame}
		\frametitle{Protokol brez zavisti za $n = 3$}
		\begin{enumerate}
			\item Igralec 1 razreže kolač na 3 dele.
			
			\item Igralec 2 ne stori nič ali obreže 1 kos.
			
			\item[] Če igralec 2 ne stori nič:
			
			\item \qquad Igralec 3 izbere kos.
			
			\item \qquad Igralec 2 izbere kos.
			
			\item \qquad Igralec 1 dobi preostali kos.
			
			\item[] Če igralec 2 obreže 1 kos:
			
			\setcounter{enumi}{2}
			
			\item \qquad Igralec 3 izbere kos.
			
			\item \qquad Igralec 2 izbere kos. Če je na voljo, mora izbrati obrezani kos.
			
			\item \qquad Igralec 1 dobi preostali kos.
		\end{enumerate}
	\end{frame}
	
	\begin{frame}
		\begin{enumerate}			
			\setcounter{enumi}{5}
			
			\item \qquad Igralec 2 ali igralec 3, ki je prejel neobrezan kos, razreže odrezke na 3 dele.
			
			\item \qquad Igralec, ki je prejel obrezan kos, izbere kos.
			
			\item \qquad Igralec 1 izbere kos.
			
			\item \qquad Igralec, ki je razrezal odrezke, dobi preostali kos.
		\end{enumerate}
	\end{frame}



	\section{Zaključek}
	
	\subsection{Viri}
	
	\begin{frame}
		\frametitle{Nalsov}
		Nekaj
	\end{frame}

\end{document}