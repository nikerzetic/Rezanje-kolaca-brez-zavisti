% !TeX spellcheck = sl_SI
\documentclass[a4paper, 12pt]{article}
\usepackage[slovene]{babel}
\usepackage[utf8]{inputenc}
\usepackage[T1]{fontenc}
\usepackage{lmodern}

\usepackage{graphicx}
\usepackage{float}

\usepackage{amssymb}
\usepackage{amsmath,amsfonts}
\usepackage{amsthm}
\usepackage{mathtools}
\usepackage{subfigure}
\usepackage{wrapfig}
\usepackage{lipsum}
\usepackage{tikz}
\usepackage{multirow}

\usepackage{array}
\usepackage{multicol}

\usepackage{booktabs}

\title{
	\textbf{Rezanje kolača brez zavisti} \\
	\large Predstavitev protokolov razdeljevanja 
}

\author{Nik Erzetič}

\newtheorem*{izrek}{Izrek}
\newtheorem*{definicija}{Definicija}
\newtheorem*{posledica}{Posledica}
\newtheorem*{trditev}{Trditev}
\newtheorem*{lema}{Lema}

\begin{document}
	
	\maketitle
	
	Uvod: zakaj pomembno, predstavil protokole za rezanje torte predstavljene v izvornem članku, definiral par pojmov, uporabno v računalništvu
	
	Kako razrezati kolač, da bo vsak otrok zadovoljen s svojim kosom? Kako razporediti hišna opravila, da se nihče ne bo pritoževal, češ da mora storiti več kot ostali? Kako razdeliti sporno ozemlje med sosednji državi? V tem članku bom podal štiri protokole, ki rešijo prva dva problema in ki so navdihnili protokole, s katerim se lahko odgovori na tretje vprašanje. Zapisal jih bom tako, kot so predstavljeni v članku An Envy-Free Cake Division Protocol \cite{brams-taylor} avtorjev Brams in Taylor. Ti protokoli so: \textit{razreži in izberi}, \textit{proporcionalni protokol za $n = 3$}, \textit{proporcionalni protokol za poljuben $n$} in \textit{protokol brez zavisti za $n = 3$}. V članku izvornem članku je opisan še \textit{protokol brez zavisti za poljuben $n$}, vendar ga ne bom podrobno opisal, ker je v mojih očeh za vsakdanje situacije nepraktičen.
	
	\begin{thebibliography}{5}
		
		\bibitem{brams-taylor}
		Steven J. Brams, Alan D. Taylor
		An Envy-Free Cake Division Protocol
		\textit{The American Mathematical Monthly} \textbf{102} (1995), 9–18
		
	\end{thebibliography}
	
\end{document}














