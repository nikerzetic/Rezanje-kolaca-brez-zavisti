% !TeX spellcheck = sl_SI
\documentclass[a4paper, 12pt]{article}
\usepackage[slovene]{babel}
\usepackage[utf8]{inputenc}
\usepackage[T1]{fontenc}
\usepackage{lmodern}

\usepackage{graphicx}
\usepackage{float}

\usepackage{amssymb}
\usepackage{amsmath,amsfonts}
\usepackage{amsthm}
\usepackage{mathtools}
\usepackage{subfigure}
\usepackage{wrapfig}
\usepackage{lipsum}
\usepackage{tikz}
\usepackage{multirow}

\usepackage{array}
\usepackage{multicol}
\usepackage{enumerate}
\usepackage{enumitem}

\usepackage{booktabs}

\title{
	\textbf{Rezanje kolača brez zavisti} \\
	\large Predstavitev protokolov razdeljevanja 
}

\author{Nik Erzetič}

\newtheorem{izrek}{Izrek}
\newtheorem{definicija}{Definicija}
\newtheorem{posledica}{Posledica}
\newtheorem{trditev}{Trditev}
\newtheorem{lema}{Lema}
\newtheorem{protokol}{Protokol}
\newenvironment{dokaz}{\paragraph{Dokaz:}}{\hfill$\square$\\}

\begin{document}
	
	\maketitle
	
	Kako razrezati kolač, da bo vsak otrok zadovoljen s svojim kosom? Kako razporediti hišna opravila, da se nihče ne bo pritoževal, češ da mora storiti več kot ostali? Kako razdeliti sporno ozemlje med sosednji državi? V tem članku bom podal štiri protokole, ki rešijo prva dva problema in ki so navdihnili protokole, s katerim se lahko odgovori na tretje vprašanje. Zapisal jih bom tako, kot so predstavljeni v članku An Envy-Free Cake Division Protocol \cite{brams-taylor} avtorjev Brams in Taylor. Ti protokoli so: \textit{razreži in izberi}, \textit{proporcionalni protokol za $n = 3$}, \textit{proporcionalni protokol za poljuben $n$} in \textit{protokol brez zavisti za $n = 3$}. V izvornem članku je opisan še \textit{protokol brez zavisti za poljuben $n$}, vendar ga ne bom podrobneje opisal, ker je v mojih očeh za vsakdanje situacije nepraktičen.
	
	Definicije in protokoli v tem članku bodo skoraj povsem enaki tistim, ki jih najdemo v izvornem delu \cite{brams-taylor}. Protokoli, dokazi in zmagovalne strategije so v njem podani hkrati, jaz pa jih bom tu ločil. Prav tako bom postopek zapisal kot psevdokodo.
	
	Preden začnem opisovati protokole, moram definirati še nekaj pojmov. Pomembno vlogo v vseh protokolih bo igrala mera. Ta pojem intuitivno razumemo - gre za način vrednotenja, ki ga uporablja vsak izmed igralcev v protokolu deljenja. Vendar je mera pojem, ki ga definiramo na merljivem prostoru. Iz tega razloga bom najprej definiral $\sigma$-algebro.
	
	\begin{definicija}
		Naj bo $X$ neprazna množica. Družino $\mathcal{M}$, $\mathcal{M} \subseteq \mathcal{P}(X)$, imenujemo \textbf{$\sigma$-algebra}, če velja:
		\begin{itemize}
			\item $X \in \mathcal{M}$
			
			\item $\forall E \in \mathcal{P}(X).\ E \in \mathcal{M} \Rightarrow E^C \in \mathcal{M}$ (zaprtost za komplemente)
			
			\item $\forall E_1, \ldots, E_n \in \mathcal{P}(X).\ E_1, \ldots, E_n \in \mathcal{M} \Rightarrow \bigcup_{i=1}^{n} E_i \in \mathcal{M}$ (zaprtost za končne unije)
		\end{itemize}
		Elementom $\sigma$-algebre rečemo \textbf{merljive množice}, paru $(X, \mathcal{M})$ pa \textbf{merljiv prostor}.
	\end{definicija}

	\begin{definicija}
		Naj bo $(X, \mathcal{M})$ merljiv prostor. Preslikava $V : \mathcal{M} \to [0, \infty]$ je \textbf{mera} na $\mathcal{M}$, če velja:
		\begin{itemize}
			\item $V(\emptyset) = 0$
			\item če je $\{E_i\}_{i=1}^{\infty}$ zaporedje paroma disjunktnih merljivih množic, je potem $\mu(\cup_{i=1}^{\infty}E_i) = \sum_{i=1}^{\infty} \mu(E_i)$ ($\mu$ je števno aditivna)
		\end{itemize}		
	\end{definicija}

	Vpeljati moram še nekaj pojmov, ki bi jih s težavo formalno definiral. Prav tako so na sledeč način definirano bolj intuitivno razumljivi.
	
	\textbf{Protokol} je interaktiven postopek, ki ga lahko zapišemo kot računalniški program in ki sodelujočim lahko postavlja vprašanja, ki spremenijo njegov končni izid. Od algoritma se razlikuje ravno po izbirah igralcev, ki vplivajo na končni izid. Tako bi bila primer algoritma \textit{for zanka}, primer protokola pa barvanje platna v barvo, ki jo izbere uporabnik.
	
	Protokol je \textbf{proporcionalen}, če za vsakega igralca obstaja strategija, ki mu bo zagotovila vsaj $\frac{1}{n}$ kolača (glede na igralčevo mero). Protokol je \textbf{brez zavisti}, če za vsakega igralca obstaja strategija, ki mu bo zagotovila kos, ki je večji ali enak ostalim kosom. 
	
	Kot sem že zapisal, bom sledeče protokole predstavil s psevdokodo. V dokazih bom moral zato le poiskati optimalno strategijo in dokazati, da vsakega igralca privede do želenega rezultata. Protokoli že po svoji definiciji igralcem ponujajo različne opcije, zato bodo dokazi v večini temeljili na obravnavi primerov. Prav tako bom v dokazih za proporcionalnost predpostavil, da mera celotnega kolača za vse igralce enaka $1$. 
	
	Preden se podam v opisovanje protokolov, bi rad namenil še nekaj besed \textbf{optimalni strategiji}. Ta vsakemu igralcu, ki se je drži, zagotovi primeren kos - proporcionalen ali brez zavisti. Če ji igralec ne sledi, bo na koncu prejel "slabši" kos, kot drugače. Z drugimi besedami - ne obstaja strategija, ki bi igralcu zagotovila boljši ko. 
	
	Zdaj pa k opisovanju protokolov. Prvi je \textit{razreži in razdeli} (cut-and-choose) za dva igralca. Ta protokol je hkrati proporcionalen in brez zavisti. Zgleda pa tako:
	
	\begin{protokol} 
		\label{razrezi&razdeli}
		Razreži in razdeli:
		\begin{enumerate}
			
			\item Igralec 1 kolač razreže na 2 dela.
			
			\item Igralec 2 izbere kos.
			
			\item Igralec 1 dobi preostali kos.
			
		\end{enumerate}
	\end{protokol}
	
	\begin{trditev}
		Protokol \ref{razrezi&razdeli} je proporcionalen in brez zavisti.
	\end{trditev}

	\begin{dokaz}
		Protokol z optimalno strategijo za oba igralca je sledeč:
		\begin{enumerate}
			
			\item Igralec 1 razreže kolač $P$ na kosa $P_1$ in $P_2$, da velja $\mu_1 (P_1) = V_1 (P_2) = \frac{1}{2}$.
			
			\item Igralec 2 izbere kos $P_{i_1}$, da velja $\mu_2 (P_{i_1}) \geq \mu_2 (P_{i_2})$, kjer sta $i_1, i_2$ različna elementa $\{1, 2\}$. Ker je vsota $\mu_2 (P_{i_1}) + \mu_2 (P_{i_2}) = 1$ in $\mu_2 (P_{i_1}) \geq \mu_2 (P_{i_2})$, je $\mu_2 (P_{i_1}) \geq \frac{1}{2}$. Torej strategija igralcu 2 zagotovi kos, ki je večji ali enak $\frac{1}{2}$ in večji ali enak ostalim kosom.
			
			\item Igralec 1 dobi preostali kos $P_{i_2}$. Ker je $\mu_1 (P_1) = \mu_1 (P_2) = \frac{1}{2}$, je $\mu_1 (P_{i_2}) \geq \frac{1}{2}$ in $\mu_1 (P_{i_2}) \geq \mu_1 (P_{i_1})$. Torej strategija igralcu 1 zagotovi kos, ki je večji ali enak $\frac{1}{2}$ in večji ali enak ostalim kosom.
			
		\end{enumerate}
	
		Torej je protokol razreži in razdeli proporcionalen in brez zavisti.
	\end{dokaz}

	Naslednji je \textit{proporcionalni protokol za $n = 3$} (proportional protocol for $n=3$). Razvil ga je Steinhauss med drugo svetovno vojno. Poteka pa tako:
	
	\begin{protokol}
		\label{proporcionalni_3}
		Proporcionalni protokol za $n = 3$:
		\begin{enumerate}
			
			\item Igralec 1 razreže kolač na 3 dele.
			
			\item Igralec 2 ne stori nič ali označi 2 kosa.
			
			\item[] Če igralec 2 ne stori nič:
		
			\setcounter{enumi}{2}
			
			\item \qquad Igralec 3 izbere kos.
			
			\item \qquad Igralec 2 izbere kos.
			
			\item \qquad Igralec 1 dobi preostali kos.
			
			\item[] Če igralec 2 označi 2 kosa:

			\setcounter{enumi}{2}
			
			\item \qquad Igralec 3 ne stori nič ali označi 2 kosa.
			
			\item[] \qquad Če igralec 3 ne stori nič:
		
			\setcounter{enumi}{3}
			
			\item \qquad \qquad Igralec 2 izbere kos.
			
			\item \qquad \qquad Igralec 3 izbere kos.
			
			\item \qquad \qquad Igralec 1 dobi preostali kos.
			
			\item[] \qquad Če igralec 3 označi 2 kosa:

			\setcounter{enumi}{3}
			
			\item \qquad \qquad Igralec 1 izbere kos, ki sta ga označila igralec 2 in igralec 3.
			
			\item \qquad \qquad Preostala kosa se združita v nov kolač.
			
			\item \qquad \qquad Protokol \textbf{razreži in izberi} med igralcem 2 in igralcem 3.
			
		\end{enumerate}
	\end{protokol}

	\begin{trditev}
		Protokol \ref{proporcionalni_3} je proporcionalen.
	\end{trditev}

	\begin{dokaz}
		Protokol \ref{proporcionalni_3} z optimalno strategijo za vse tri igralce je sledeč:
		\begin{enumerate}
			
			\item Igralec 1 razreže kolač $P$ na kose $P_1$, $P_2$ in $P_3$, da velja $\mu_1 (P_1) = \mu_1 (P_2) = \mu_1 (P_3) = \frac{1}{3}$.
			
			\item Igralec 2 ne stori nič, če obstajata različna indeksa $i_1, i_2$ iz $\{1, 2, 3\}$, da velja $\mu_2 (P_{i_1}), \mu_2 (P_{i_2}) \geq \frac{1}{3}$, ali označi kosa $P_{i_1}$ in $P_{i_2}$, če sta $\mu_2 (P_{i_1}), \mu_2 (P_{i_2}) < \frac{1}{3}$ in $i_1$, $i_2$ različna indeksa.
			
			\item[] Predpostavimo, da je $\mu_2 (P_{i_1}), \mu_2 (P_{i_2}) \geq \frac{1}{3}$. Igralec 2 ne stori nič.
			
			\setcounter{enumi}{2}
			
			\item \qquad Igralec 3 izbere kos $P_{j_3}$, za katerega velja $\mu_3 (P_{j_3}) \geq \frac{1}{3}$. Tak kos obstaja, ker je $\mu_3 (P_{j_1}) + \mu_3 (P_{j_2}) + \mu_3 (P_{j_3}) = 1$ in so na voljo še vsi trije kosi. Torej strategija igralcu 3 zagotovi kos, ki je večji ali enak $\frac{1}{3}$.
			
			\item \qquad Igralec 2 izbere kos $P_{j_2}$ iz $\{P_{i_1}, P_{i_2}\}$. Vsaj eden izmed teh kosov je še na voljo, saj igralec 3 izbral le en kos. Po predpostavki za kos $P_{j_2}$ velja $\mu_2 (P_{j_2}) \geq \frac{1}{3}$. Torej strategija igralcu 2 zagotovi kos, ki je večji ali enak $\frac{1}{3}$.
			
			\item \qquad Igralec 1 dobi preostali kos $P_{j_1}$. Ker je igralec 1 kolač razrezal tako, da velja $\mu_1 (P_1) = \mu_1 (P_2) = \mu_1 (P_3) = \frac{1}{3}$, mu strategija zagotovi kos, ki je večji ali enak $\frac{1}{3}$. 
			
			\item[] Predpostavimo, da je $\mu_2 (P_{i_1}), \mu_2 (P_{i_2}) < \frac{1}{3}$. Igralec 2 označi kosa $P_{i_1}$ in $P_{i_2}$. 
			
			\setcounter{enumi}{2}
			
			\item \qquad Igralec 3 ne stori nič, če obstajata različna indeksa $j_1, j_2$ iz $\{1, 2, 3\}$, da velja $\mu_3 (P_{j_1}), \mu_3 (P_{j_2}) \geq \frac{1}{3}$, ali označi kosa $P_{j_1}$ in $P_{j_2}$, če sta $\mu_3 (P_{j_1}), \mu_3 (P_{j_2}) < \frac{1}{3}$ in $i_1$, $i_2$ različna indeksa.
			
			\item[] \qquad Predpostavimo, da je $\mu_3 (P_{j_1}), \mu_3 (P_{j_2}) \geq \frac{1}{3}$. Igralec 3 ne stori nič.
			
			\setcounter{enumi}{3}
			
			\item \qquad \qquad Igralec 2 izbere neoznačen kos $P_{i_3}$.
			
			\item \qquad \qquad Igralec 3 izbere kos $P_{j_2}$ iz $\{P_{i_1}, P_{i_2}\}$. Vsaj eden izmed teh kosov je še na voljo, saj igralec 3 izbral le en kos. Po predpostavki za kos $P_{j_2}$ velja $\mu_2 (P_{j_2}) \geq \frac{1}{3}$. Torej strategija igralcu 2 zagotovi kos, ki je večji ali enak $\frac{1}{3}$.
			
			\item \qquad \qquad Igralec 1 dobi preostali kos $P_{j_1}$. Ker velja $\mu_1 (P_1) = \mu_1 (P_2) = \mu_1 (P_3) = \frac{1}{3}$, je $\mu_1 (P_{j_1}) \geq \frac{1}{3}$. Torej strategija igralcu 1 zagotovi kos, ki je večji ali enak $\frac{1}{3}$.
			
			\item[] \qquad Predpostavimo, da je $\mu_3 (P_{j_1}), \mu_3 (P_{j_2}) < \frac{1}{3}$. Igralec 3 označi kosa $P_{j_1}$ in $P_{j_2}$.
			
			\setcounter{enumi}{3}
			
			\item \qquad \qquad Igralec 1 izbere kos $P_{k_1}$ iz $\{P_{i_1}, P_{i_2}\} \cap \{P_{j_1}, P_{j_2}\}$. Ker velja $\mu_1 (P_1) = \mu_1 (P_2) = \mu_1 (P_3) = \frac{1}{3}$, je $\mu_1 (P_{k_1}) \geq \frac{1}{3}$. Torej strategija igralcu 1 zagotovi kos, ki je večji ali enak $\frac{1}{3}$.
			
			\item \qquad \qquad Preostala kosa se združita v nov kolač.
			
			\item \qquad \qquad Preostala kosa se združita v nov kolač, ki se med igralca 1 in igralca 2 razdeli po protokolu \textbf{razreži in razdeli}. Optimalna strategija zanj obema zagotovo kos, ki je večji ali enak $\frac{1}{3}$.
			
		\end{enumerate}
		Torej je proporcionalni protokol za $n = 3$ proporcionalen.
	\end{dokaz}

	Proporcionalen protokol za poljuben $n$ sta kmalu po Steinhausovem podala Banach in Knaster. Izgledato tako:

	\begin{protokol}
		\label{proporcionalni_n}
		Proporcionalni protokol za poljuben $n$:
		\begin{enumerate}
			
			\item Igralec 1 odreže kos od kolača.
			
			\item \begin{enumerate}
							
				\item[(1.)] Igralec 2 ne stori nič ali obreže odrezani kos.
				
				\item[] $\vdots$
				
				\item[(i.)] Igralec i ne stori nič ali obreže odrezani kos.
				
				\item[] $\vdots$
				
				\item[(n.)] Igralec n ne stori nič ali obreže odrezani kos.
				
			\end{enumerate}
		
			\item Zadnji igralec, ki je obrezal kos, ali igralec 1, če nihče ni obrezal kosa, prejme ta kos. 
			
			\item Odrezke se združijo s kolačem.
			
			\item Koraki 1. - 4. se ponavljajo, dokler ne ostaneta dva igralca.
			
			\item Protokol \textbf{razreži in izberi} med preostalima igralcem.
			
		\end{enumerate}
	\end{protokol}

	\begin{trditev}
		Protokol \ref{proporcionalni_n} je proporcionalen.
	\end{trditev}

	\begin{dokaz}
		Protokol \ref{proporcionalni_n} z optimalno strategijo za vseh n igralcev je sledeč:
		\begin{enumerate}
			
			\item Igralec 1 odreže kos $P_1$ od kolača $P$, da velja $\mu_1 (P_1) = \frac{1}{n}$.
			
			\item \begin{enumerate}
				
				\item[(1.)] Igralec 2 ne stori nič, če je $\mu_2 (P_1) \leq \frac{1}{n}$, ali obreže odrezani kos, če je $\mu_2 (P_1) > \frac{1}{n}$, da dobi kos $P_2$, za katerega velja $\mu_2 (P_2) = \frac{1}{n}$.
				
				\item[] $\vdots$
				
				\item[(i.)] Igralec i ne stori nič, če je $\mu_i (P_{i-1}) \leq \frac{1}{n}$, ali obreže odrezani kos, če je $\mu_i (P_{i-1}) > \frac{1}{n}$, da dobi kos $P_i$, za katerega velja $\mu_i (P_i) = \frac{1}{n}$.
				
				\item[] $\vdots$
				
				\item[(n.)] Igralec n ne stori nič, če je $\mu_n (P_{n-1}) \leq \frac{1}{n}$, ali obreže odrezani kos, če je $\mu_n (P_{n-1}) > \frac{1}{n}$, da dobi kos $P_n$, za katerega velja $\mu_n (P_n) = \frac{1}{n}$.
				
			\end{enumerate}
			
			\item Naj bo igralec i tisti, ki je zadnji obrezal kos. Velja $\mu_i (P_i) = \frac{1}{n}$. Torej strategija igralcu i zagotovi kos, ki je večji ali enak $\frac{1}{n}$.
			
			\item Odrezke se združijo s kolačem.
			
			\item Koraki 1. - 4. se ponavljajo, dokler ne ostaneta dva igralca. V vsakem koraku igralec j dobi kos, za katerega velja $\mu_j (P_j) = \frac{1}{n}$. Pri tem je potrebno dokazati še, da vsak naslednji igralec dobi $\frac{1}{n}$. To pokaže preprost račun:
			$$
			\frac{1 - \frac{1}{n}}{n - 1} = \frac{\frac{n}{n} - \frac{1}{n}}{n - 1} = \frac{n - 1}{n (n - 1)} = \frac{1}{n}
			$$
			Torej optimalna strategija igralcu j zagotovi kos, ki je večji ali enak $\frac{1}{n}$.
			
			\item Protokol \textbf{razreži in izberi} med preostalima igralcem vsakemu izmed njiju zagotovi del, ki je večji ali enak $\frac{1}{n}$.
			
		\end{enumerate}
		Torej je proporcionalni protokol za poljuben $n$ proporcionalen.
	\end{dokaz}

	Zadnji protokol, ki ga bom predstavil je protokol brez zavisti za $n = 3$. Prva sta ga predstavila Selfridge in Conway ter je sledeč:

	\begin{protokol}
		\label{brez_zavisti}
		Protokol brez zavisti za $n = 3$
		\begin{enumerate}
			
			\item Igralec 1 razreže kolač na 3 dele.
			
			\item Igralec 2 ne stori nič ali obreže 1 kos.
			
			\item[] Če igralec 2 ne stori nič:
			
			\item \qquad Igralec 3 izbere kos.
			
			\item \qquad Igralec 2 izbere kos.
			
			\item \qquad Igralec 1 dobi preostali kos.
			
			\item[] Če igralec 2 obreže 1 kos:
			
			\setcounter{enumi}{2}
			
			\item \qquad Igralec 3 izbere kos.
			
			\item \qquad Igralec 2 izbere kos. Če je na voljo, mora izbrati obrezani kos.
			
			\item \qquad Igralec 1 dobi preostali kos.
			
			\item \qquad Igralec 2 ali igralec 3, ki je prejel neobrezan kos, razreže odrezke na 3 dele.
			
			\item \qquad Igralec, ki je prejel obrezan kos, izbere kos.
			
			\item \qquad Igralec 1 izbere kos.
			
			\item \qquad Igralec, ki je razrezal odrezke, dobi preostali kos.
			
		\end{enumerate}
	\end{protokol}
	
	\begin{trditev}
		Protokol \ref{brez_zavisti} je brez zavisti.
	\end{trditev}

	\begin{dokaz}
		Protokol \ref{brez_zavisti} z optimalno strategijo za vse tri igralce je sledeč:
		\begin{enumerate}
			
			\item Igralec 1 razreže kolač na dele $P_1$, $P_2$ in $P_3$, za katere velja $\mu_1 (P_1) = \mu_2 (P_2) = \mu_3 (P_3)$.
			
			\item Igralec 2 ne stori nič, če obstajata kosa $P_{i_1}$ in $P_{i_2}$, da velja $\mu_2 (P_{i_1}), \mu_2 (P_{i_2}) \geq \mu_2 (P_1), \mu_2 (P_2), \mu_2 (P_3)$, ali obreže kos $P_{i_1}$, če velja $\mu_2 (P_{i_1}) > \mu_2 (P_{i_2}), \mu_2 (P_{i_3})$.
			
			\item[] Predpostavimo, da velja $\mu_2 (P_{i_1}), \mu_2 (P_{i_2}) \geq \mu_2 (P_1), \mu_2 (P_2), \mu_2 (P_3)$. Igralec 2 ne stori nič.
			
			\item \qquad Igralec 3 izbere kos $P_{j_3}$, za katerega velja $\mu_3 (P_{j_3}) \geq \mu_3 (P_1), \mu_3 (P_2), \mu_3 (P_3)$. Torej optimalna strategija igralcu 3 zagotovi kos, ki večji ali enak ostalim kosom.
			
			\item \qquad Igralec 2 izbere kos $P_{j_2}$, za katerega velja $\mu_2 (P_{j_2}) \geq \mu_2 (P_1), \mu_2 (P_2), \mu_2 (P_3)$. Po predpostavki obstajata dva taka kosa in igralec 3 je izbral največ enega. Torej optimalna strategija igralcu 2 zagotovi kos, ki večji ali enak ostalim kosom.
			
			\item \qquad Igralec 1 dobi preostali kos $P_{j_1}$. Ker za vse tri kose velja $\mu_1 (P_1) = \mu_2 (P_2) = \mu_3 (P_3)$, optimalna strategija igralcu 1 zagotovi kos, ki je večji ali enak ostalim kosom.
			
			\item[] Predpostavimo, da velja $\mu_2 (P_{i_1}) > \mu_2 (P_{i_2}), \mu_2 (P_{i_3})$ Igralec 2 obreže kos $P_{i_1}$ tako, da sta vsaj dva kosa večja ali enaka ostalim. Odrezke označimo z $L$.
			
			\setcounter{enumi}{2}
			
			\item \qquad Igralec 3 izbere kos $P_{j_3}$, za katerega velja $\mu_3 (P_{j_3}) \geq \mu_3 (P_{i_1}), \mu_3 (P_{i_2}), \mu_3 (P_{i_3})$. Tak kos obstaja, ker so na voljo še vsi kosi.
			
			\item \qquad Igralec 2 izbere kos $P_{j_2}$. Če je na voljo, mora izbrati obrezani kos. Zanj velja $\mu_2 (P_{j_2}) \geq \mu_2 (P_{i_1}), \mu_2 (P_{i_2}), \mu_2 (P_{i_3})$, ker je enega od kosov obrezal tako, da sta vsaj dva kosa večja ali enaka ostalim, in ker je igralec 3 izbral največ enega izmed teh kosov.
			
			\item \qquad Igralec 1 dobi preostali kos $P_{j_1}$. Ta je večji ali enak ostalim kosom, ker velja $\mu_1 (P_1) = \mu_2 (P_2) = \mu_3 (P_3)$ in ker je obrezani kos manjši od prvotnega.
			
			\item \qquad Ker se ne glede na to, kdo izbere obrezan kos, lahko brez škode za splošnost predpostavimo, da ga je izbral igralec 3. Igralec 2 potem razreže odrezke $L$ na kose $L_1$, $L_2$ in $L_3$.
			
			\item \qquad Igralec 3 izbere kos $L_{k_3}$, za katerega velja $\mu_N (L_{k_3}) \geq \mu_N (L_1), \mu_N (L_2), \mu_N (L_3)$. Ker je igralec izbral kos $P_{j_3}$, ki je večji ali enak ostalim, in odrezke $L_{k_3}$, ki so večji ali enaki ostalim, je njegov skupen kos $P_{j_3} \cup L_{k_3}$ večji ali enak ostalim kosom. Torej mu optimalna strategija zagotovi kos, ki je večji ali enak ostalim kosom.
			
			\item \qquad Igralec 1 izbere kos $L_{k_1}$, ki je večji ali enak drugemu preostalemu kosu $L_{k_2}$. Ker je $\mu_1 (P_1) = \mu_2 (P_2) = \mu_3 (P_3)$ in ker je igralec 3 vzel obrezan kos, je $P_{j_1} > P_{j_3} \cup L_{k_3}$. Torej optimalna strategija igralcu 1 zagotovi kos, ki je večji ali enak ostalim kosom.
			
			\item \qquad Igralec 2 dobi preostali kos $L_{k_2}$. Ker je $P_{j_2}$ večji ali enak ostalim kosom in $L_{k_2}$ enak ostalim odrezkom, je njegov skupen kos $P_{j_3} \cup L_{k_3}$ večji ali enak ostalim kosom. Torej optimalna strategija igralcu 2 zagotovi kos, ki je večji ali enak ostalim kosom.
			
		\end{enumerate}
		Torej je protokol brez zavisti za $n = 3$ brez zavisti.
	\end{dokaz}

	Na koncu želim še odgovoriti na vprašanja, ki sem jih zastavil na začetku članka. Kolač lahko med otroke pravično razdelimo z enim izmed protokolov za primerno število oseb. Najboljši za to bi bil zadnji protokol, pod pogojem, da kolač delimo med tri otroke. Prav tako lahko hišna opravila razdelimo z enim izmed zgornjih protokolov, če le zamenjamo \textit{večje ali enako} z \textit{manjše ali enako} in \textit{večje} z \textit{manjše}. Problem delitve zemlje je prvi zastavil Ted Hill leta 1983 \cite{hill}, rešitev pa je podal Anatole Beck \cite{beck} nekaj let za tem .
	
	\pagebreak
	
	\begin{thebibliography}{5}
		
		\bibitem{brams-taylor}
		Steven J. Brams, Alan D. Taylor
		An Envy-Free Cake Division Protocol
		\textit{The American Mathematical Monthly} \textbf{102} (1995), 9–18
		
		\bibitem{hill}
		T. P. Hill 
		Determining a Fair Border
		\textit{The American Mathematical Monthly} \textbf{90} (1983), 438–442
		
		\bibitem{beck}
		A. Beck 
		Constructing a Fair Border
		\textit{The American Mathematical Monthly} \textbf{94} (1987), 157–162
		
		\bibitem{sigma}
		\textbf{Sigma-algebra}, dostop: \textit{http://wiki.fmf.uni-lj.si/wiki/Merljiv\_prostor}		
		
		\bibitem{mera}
		\textbf{Mera}, dostop: \textit{http://wiki.fmf.uni-lj.si/wiki/Mera}
		
	\end{thebibliography}
	
\end{document}














