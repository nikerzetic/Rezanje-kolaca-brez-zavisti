% !TeX spellcheck = sl_SI
\documentclass[a4paper, 12pt]{article}
\usepackage[slovene]{babel}
\usepackage[utf8]{inputenc}
\usepackage[T1]{fontenc}
\usepackage{lmodern}

\usepackage{graphicx}
\usepackage{float}

\usepackage{amssymb}
\usepackage{amsmath,amsfonts}
\usepackage{amsthm}
\usepackage{mathtools}
\usepackage{subfigure}
\usepackage{wrapfig}
\usepackage{lipsum}
\usepackage{tikz}
\usepackage{multirow}

\usepackage{array}
\usepackage{multicol}
\usepackage{enumerate}
\usepackage{enumitem}

\usepackage{booktabs}

\title{
	\textbf{Rezanje kolača brez zavisti} \\
	\large Predstavitev protokolov razdeljevanja 
}

\author{Nik Erzetič}

\newtheorem{izrek}{Izrek}
\newtheorem{definicija}{Definicija}
\newtheorem{posledica}{Posledica}
\newtheorem{trditev}{Trditev}
\newtheorem{lema}{Lema}
\newtheorem{protokol}{Protokol}
\newenvironment{dokaz}{\paragraph{Dokaz:}}{\hfill$\square$\\}

\begin{document}
	
	\maketitle
	
	Uvod: zakaj pomembno, predstavil protokole za rezanje torte predstavljene v izvornem članku, definiral par pojmov, uporabno v računalništvu
	
	Kako razrezati kolač, da bo vsak otrok zadovoljen s svojim kosom? Kako razporediti hišna opravila, da se nihče ne bo pritoževal, češ da mora storiti več kot ostali? Kako razdeliti sporno ozemlje med sosednji državi? V tem članku bom podal štiri protokole, ki rešijo prva dva problema in ki so navdihnili protokole, s katerim se lahko odgovori na tretje vprašanje. Zapisal jih bom tako, kot so predstavljeni v članku An Envy-Free Cake Division Protocol \cite{brams-taylor} avtorjev Brams in Taylor. Ti protokoli so: \textit{razreži in izberi}, \textit{proporcionalni protokol za $n = 3$}, \textit{proporcionalni protokol za poljuben $n$} in \textit{protokol brez zavisti za $n = 3$}. V izvornem članku je opisan še \textit{protokol brez zavisti za poljuben $n$}, vendar ga ne bom podrobneje opisal, ker je v mojih očeh za vsakdanje situacije nepraktičen.
	
	Definicije in protokoli v tem članku bodo skoraj povsem enaki tistim, ki jih najdemo v izvornem delu \cite{brams-taylor}. Protokoli, dokazi in zmagovalne strategije so v njem podani hkrati, jaz pa jih bom tu ločil. Prav tako bom postopek zapisal kot psevdokodo.
	
	Preden začnem opisovati protokole, moram definirati še nekaj pojmov. Prva od teh je - zdaj že velikokrat omenjena beseda - protokol. Sledita še dve definiciji o lastnostih protokolov - proporcionalnost in brez zavisti - ki sem ju prav tako že omenil v uvodnem odstavku.
	
	\begin{definicija}
		Naj bo $X$ neprazna množica. Družino $\mathcal{M}$, $\mathcal{M} \subseteq \mathcal{P}(X)$, imenujemo \textbf{$\sigma$-algebra}, če velja:
		\begin{itemize}
			\item $X \in \mathcal{M}$
			
			\item $\forall E \in \mathcal{P}(X).\ E \in \mathcal{M} \Rightarrow E^C \in \mathcal{M}$ (zaprtost za komplemente)
			
			\item $\forall E_1, \ldots, E_n \in \mathcal{P}(X).\ E_1, \ldots, E_n \in \mathcal{M} \Rightarrow \bigcup_{i=1}^{n} E_i \in \mathcal{M}$ (zaprtost za končne unije)
		\end{itemize}
		Elementom $\sigma$-algebre rečemo \textbf{merljive množice}, paru $(X, \mathcal{M})$ pa \textbf{merljiv prostor}.
	\end{definicija}

	\begin{definicija}
		Naj bo $(X, \mathcal{M})$ merljiv prostor. Preslikava $V : \mathcal{M} \to [0, \infty]$ je \textbf{mera} na $\mathcal{M}$, če velja:
		\begin{itemize}
			\item $V(\emptyset) = 0$
			\item če je $\{E_i\}_{i=1}^{\infty}$ zaporedje paroma disjunktnih merljivih množic, je potem $\mu(\cup_{i=1}^{\infty}E_i) = \sum_{i=1}^{\infty} \mu(E_i)$ ($\mu$ je števno aditivna)
		\end{itemize}		
	\end{definicija}

	Vpeljati moram še nekaj pojmov, ki bi jih s težavo formalno definiral. Prav tako so na sledeč način definirano bolj intuitivno razumljivi.
	
	\textbf{Protokol} je interaktiven postopek, ki ga lahko zapišemo kot računalniški program in ki sodelujočim lahko postavlja vprašanja, ki spremenijo njegov končni izid. Od algoritma se razlikuje ravno po izbirah igralcev, ki vplivajo na končni izid. Tako bi bila primer algoritma \textit{for zanka}, primer protokola pa barvanje platna v barvo, ki jo izbere uporabnik.
	
	Protokol je \textbf{proporcionalen}, če za vsakega igralca obstaja strategija, ki mu bo zagotovila vsaj $\frac{1}{n}$ kolača (glede na lasten kriterij). Protokol je \textbf{brez zavisti}, če za vsakega igralca obstaja strategija, ki mu bo zagotovila kos, ki je večji ali enak ostalim kosom. 
	
	Kot sem že zapisal, bom sledeče protokole predstavil s psevdokodo. V dokazih bom moral zato le poiskati optimalno strategijo in dokazati, da vsakega igralca privede do želenega rezultata. Protokoli že po svoji definiciji igralcem ponujajo različne opcije, zato bodo dokazi v večini temeljili na obravnavi primerov.
	
	Prvi protokol je \textit{razreži in razdeli} (cut-and-choose) za dva igralca. Ta protokol je hkrati proporcionalen in brez zavisti. Zgleda pa tako:
	
	\begin{protokol} 
		\label{razrezi&razdeli}
		Razreži in razdeli:
		\begin{enumerate}
			
			\item Igralec 1 kolač razreže na 2 dela.
			
			\item Igralec 2 izbere kos.
			
			\item Igralec 1 dobi preostali kos.
			
		\end{enumerate}
	\end{protokol}
	
	\begin{trditev}
		Protokol \ref{razrezi&razdeli} je proporcionalen in brez zavisti.
	\end{trditev}

	\begin{dokaz}
		Protokol z optimalno strategijo za oba igralca je sledeč:
		\begin{enumerate}
			
			\item Igralec 1 razreže kolač $P$ na kosa $P_1$ in $P_2$, da velja $\mu_1 (P_1) = V_1 (P_2) = \frac{1}{2}$.
			
			\item Igralec 2 izbere kos $P_{i_1}$, da velja $\mu_2 (P_{i_1}) \geq \mu_2 (P_{i_2})$, kjer sta $i_1, i_2$ elementa $\{1, 2\}$ in $i_1 \neq i_2$. Ker je vsota $\mu_2 (P_{i_1}) + \mu_2 (P_{i_2} = 1$ in $\mu_2 (P_{i_1}) \geq \mu_2 (P_{i_2})$, je $\mu_2 (P_{i_1}) \geq \frac{1}{2}$. Torej strategija igralcu 2 zagotovi kos, ki je večji ali enak $\frac{1}{2}$ in večji ali enak ostalim kosom.
			
			\item Igralec 1 dobi preostali kos $P_{i_2}$. Ker je $\mu_1 (P_1) = \mu_1 (P_2) = \frac{1}{2}$, je $\mu_1 (P_{i_2}) \geq \frac{1}{2}$ in $\mu_1 (P_{i_2}) \geq \mu_1 (P_{i_1})$. Torej strategija igralcu 1 zagotovi kos, ki je večji ali enak $\frac{1}{2}$ in večji ali enak ostalim kosom.
			
		\end{enumerate}
	
		Torej je protokol razreži in razdeli proporcionalen in brez zavisti.
	\end{dokaz}

	Naslednji je \textit{proporcionalni protokol za $n = 3$} (proportional protocol for $n=3$). Razvil ga je Steinhauss med drugo svetovno vojno. Poteka pa tako:
	
	\begin{protokol}
		\label{proporcionalni_3}
		Proporcionalni protokol za $n = 3$:
		\begin{enumerate}
			
			\item Igralec 1 razreže kolač na 3 dele.
			
			\item Igralec 2 ne stori nič ali označi 2 kosa.
			
			\item[] Če igralec 2 ne stori nič:
		
			\setcounter{enumi}{2}
			
			\item \qquad Igralec 3 izbere kos.
			
			\item \qquad Igralec 2 izbere kos.
			
			\item \qquad Igralec 1 dobi preostali kos.
			
			\item[] Če igralec 2 označi 2 kosa:

			\setcounter{enumi}{2}
			
			\item \qquad Igralec 3 ne stori nič ali označi 2 kosa.
			
			\item[] \qquad Če igralec 3 ne stori nič:
		
			\setcounter{enumi}{3}
			
			\item \qquad \qquad Igralec 2 izbere kos.
			
			\item \qquad \qquad Igralec 3 izbere kos.
			
			\item \qquad \qquad Igralec 1 dobi preostali kos.
			
			\item[] \qquad Če igralec 3 označi 2 kosa:

			\setcounter{enumi}{3}
			
			\item \qquad \qquad Igralec 1 izbere kos, ki sta ga označila igralec 2 in igralec 3
			
			\item \qquad \qquad Preostala kosa se združita v nov kolač
			
			\item \qquad \qquad Protokol \textbf{razreži in izberi} med igralcem 2 in igralcem 3
			
		\end{enumerate}
	\end{protokol}

	\begin{trditev}
		Protokol \ref{proporcionalni_3} je proporcionalen.
	\end{trditev}

	\begin{dokaz}
		Protokol z optimalno strategijo za vse tri igralce je sledeč:
	\end{dokaz}

	\begin{protokol}
		\label{proporcionalni_n}
		Proporcionalni protokol za poljuben $n$:
		\begin{enumerate}
			
			\item Igralec 1 odreže kos od kolača.
			
			\item \begin{enumerate}
							
				\item[(1.)] Igralec 2 ne stori nič ali obreže odrezani kos.
				
				\item[] $\vdots$
				
				\item[(i.)] Igralec i ne stori nič ali obreže odrezani kos.
				
				\item[] $\vdots$
				
				\item[(n.)] Igralec n ne stori nič ali obreže odrezani kos.
				
			\end{enumerate}
		
			\item Zadnji igralec, ki je obrezal kos, ali igralec 1, če nihče ni obrezal kosa, prejme ta kos.
			
			\item Odrezke se združijo s kolačem.
			
			\item Koraki 1. - 4. se ponavljajo, dokler ne ostaneta dva igralca.
			
			\item Protokol \textbf{razreži in izberi} med preostalima igralcem.
			
		\end{enumerate}
	\end{protokol}

	\begin{trditev}
		Protokol \ref{proporcionalni_n} je proporcionalen.
	\end{trditev}

	\begin{protokol}
		\label{brez_zavisti}
		Protokol brez zavisti za $n = 3$
		\begin{enumerate}
			
			\item Igralec 1 razreže kolač na 3 dele.
			
			\item Igralec 2 ne stori nič ali obreže 1 kos.
			
			\item[] Če igralec 2 ne stori nič:
			
			\item \qquad Igralec 3 izbere kos.
			
			\item \qquad Igralec 2 izbere kos.
			
			\item \qquad Igralec 1 dobi preostali kos.
			
			\item[] Če igralec 2 obreže 1 kos:
			
			\setcounter{enumi}{2}
			
			\item \qquad Igralec 3 izbere kos.
			
			\item \qquad Igralec 2 izbere kos. Če je na voljo, mora izbrati obrezani kos.
			
			\item \qquad Igralec 1 dobi preostali kos.
			
			\item \qquad Igralec 2 ali igralec 3, ki je prejel neobrezan kos, razreže odrezke na 3 dele.
			
			\item \qquad Igralec, ki je prejel obrezan kos, izbere kos.
			
			\item \qquad Igralec 1 izbere kos.
			
			\item \qquad Igralec, ki je razrezal odrezke, dobi preostali kos.
			
		\end{enumerate}
	\end{protokol}
	
	\begin{trditev}
		Protokol \ref{brez_zavisti} je brez zavisti.
	\end{trditev}
	
	\pagebreak
	
	\begin{thebibliography}{5}
		
		\bibitem{brams-taylor}
		Steven J. Brams, Alan D. Taylor
		An Envy-Free Cake Division Protocol
		\textit{The American Mathematical Monthly} \textbf{102} (1995), 9–18
		
	\end{thebibliography}
	
\end{document}














